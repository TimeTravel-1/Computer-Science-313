\documentclass[11pt]{article}

\usepackage[margin=1in]{geometry}
\usepackage{mathpazo}
\usepackage{amssymb}
\usepackage{amsmath}
\usepackage{fancyhdr}

%-----------------------------------------------------------------------------%
% Macros:
%-----------------------------------------------------------------------------%

\newcommand{\op}[1]{\operatorname{#1}}
\newcommand{\tinyspace}{\mspace{1mu}}
\newcommand{\rev}{\textup{\tiny R}}

\newcommand{\abs}[1]{\lvert #1 \rvert}
\newcommand{\bigabs}[1]{\bigl\lvert #1 \bigr\rvert}
\newcommand{\Bigabs}[1]{\Bigl\lvert #1 \Bigr\rvert}
\newcommand{\biggabs}[1]{\biggl\lvert #1 \biggr\rvert}
\newcommand{\Biggabs}[1]{\Biggl\lvert #1 \Biggr\rvert}

\renewcommand{\natural}{\mathbb{N}}

\newenvironment{mylist}[1]{\begin{list}{}{
	\setlength{\leftmargin}{#1}
	\setlength{\rightmargin}{0mm}
	\setlength{\labelsep}{2mm}
	\setlength{\labelwidth}{8mm}
	\setlength{\itemsep}{0mm}}}
	{\end{list}}

\newcounter{questioncounter}
\newenvironment{question}{
  \begin{mylist}{\parindent}
  \item[\stepcounter{questioncounter}
    \thequestioncounter.]}{
\end{mylist}}

\newenvironment{solution}{
  \begin{trivlist}
  \item {\bf Solution.}}{
\end{trivlist}}

%=============================================================================%
               
\begin{document}

\pagestyle{plain}
\thispagestyle{fancy}
\rhead{\footnotesize Spring 2016}
\lhead{\footnotesize CS 360 Introduction to the Theory of Computing}
\cfoot{\thepage} 
\renewcommand{\headrulewidth}{0pt}
\renewcommand{\footrulewidth}{0pt}

\begin{center}
  \rule{0mm}{9mm}
  {\Large\bf Assignment 2}\\[2mm]
  Due: Friday, June 10 at 4:00pm
\end{center}

For any solutions to the problems below that include the specification of a
CFG (whether it is to prove that a language is context-free or because you were
asked explicitly for a CFG), you do \underline{not} need to give a formal proof
of correctness for the CFG.
However, if it is too difficult to verify the correctness of your CFGs,
then you may lose points---so please aim to make them as simple and clear as
possible, and feel free to include short explanations if you believe it will
be helpful.

%-----------------------------------------------------------------------------%

\begin{question}[6 points]
  For each of the following languages, give a CFG that generates the language:
  
  \begin{enumerate}
  \item[(a)]
    $\bigl\{w\in\{0,1\}^{\ast}\,:\,\abs{w}_0 = 2\abs{w}_1\bigr\}$.
  \item[(b)]
    $\overline{\textup{PAL}}$, where $\textup{PAL}$ is the language of
    palindromes (as defined in Lecture~7).
  \item[(c)]
    $\overline{\textup{BAL}}$, where $\textup{BAL}$ is the language of
    balanced parentheses (also as defined in Lecture~7).
  \end{enumerate}

  (For part (a) and for question 4 below, $\abs{w}_\sigma$ is defined as
  the number of times the symbol~$\sigma$ appears in $w$.)
\end{question}

%-----------------------------------------------------------------------------%

\begin{question}[6 points]
  Let $\Sigma = \{0,1\}$ and let $A\subseteq\Sigma^{\ast}$ be a context-free
  language.
  Prove that the following languages are also context-free:
  
  \begin{enumerate}
  \item[(a)]
    $B = \bigl\{ uv \,:\,\;\text{$u,v\in\Sigma^{\ast}$ and there exists
    $\sigma\in\Sigma$ such that $u \sigma v \in A$}\bigr\}$.

  \item[(b)]
    $C = \bigl\{ u\sigma v \,:\,\;\text{$u,v\in\Sigma^{\ast}$, 
    $\sigma\in\Sigma$, and $u v \in A$}\bigr\}$.
  \end{enumerate}
  
  In other words, $B$ is the language of all strings you can obtain by choosing
  a string from $A$ and removing exactly one symbol from that string, while $C$
  is the language of all strings you can obtain by choosing a string from $A$
  and inserting exactly one additional symbol from $\Sigma$ anywhere into that
  string.
\end{question}

%-----------------------------------------------------------------------------%

\begin{question}[6 points]
  Let $\Sigma = \{0,1\}$ and let $A\subseteq\Sigma^{\ast}$ be a regular
  language.
  Prove that the following languages are context-free:
  
  \begin{enumerate}
  \item[(a)]
    $B = \bigl\{w w^{\rev}\,:\,w\in A\bigr\}$.

  \item[(b)]
    $C = \bigl\{u v^{\rev}\,:\,u,v\in A,\;\abs{u} = \abs{v}\bigr\}$.
    
  \item[(c)]
    $D = \bigl\{u1v\,:\,u,v\in\Sigma^{\ast},\;\abs{u}=\abs{v},\;uv\in
    A\bigr\}$.
  \end{enumerate}
\end{question}

%-----------------------------------------------------------------------------%

\begin{question}[6 points]
  Prove that the following language is not context-free:
  \[
  A = \big\{w\in\{0,1,2\}^{\ast}\,:\,\abs{w}_0 \leq \abs{w}_1 \leq \abs{w}_2
  \bigr\}.
  \]
\end{question}

%-----------------------------------------------------------------------------%

\begin{question}[1 point]
  For each of the questions above, list the full name of each of your 360
  classmates with whom you worked on that question.
  (If you didn't work with anyone, that is fine: just indicate that you worked
  alone.)
\end{question}

%-----------------------------------------------------------------------------%

\end{document}
