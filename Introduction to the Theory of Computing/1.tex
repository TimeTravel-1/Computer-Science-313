\documentclass[11pt]{article}

\usepackage[margin=1in]{geometry}
\usepackage{mathpazo}
\usepackage{amssymb}
\usepackage{amsmath}
\usepackage{fancyhdr}

%-----------------------------------------------------------------------------%
% Macros:
%-----------------------------------------------------------------------------%

\newcommand{\op}[1]{\operatorname{#1}}
\newcommand{\tinyspace}{\mspace{1mu}}

\newcommand{\abs}[1]{\lvert #1 \rvert}
\newcommand{\bigabs}[1]{\bigl\lvert #1 \bigr\rvert}
\newcommand{\Bigabs}[1]{\Bigl\lvert #1 \Bigr\rvert}
\newcommand{\biggabs}[1]{\biggl\lvert #1 \biggr\rvert}
\newcommand{\Biggabs}[1]{\Biggl\lvert #1 \Biggr\rvert}

\renewcommand{\natural}{\mathbb{N}}

\newenvironment{mylist}[1]{\begin{list}{}{
	\setlength{\leftmargin}{#1}
	\setlength{\rightmargin}{0mm}
	\setlength{\labelsep}{2mm}
	\setlength{\labelwidth}{8mm}
	\setlength{\itemsep}{0mm}}}
	{\end{list}}

\newcounter{questioncounter}
\newenvironment{question}{
  \begin{mylist}{\parindent}
  \item[\stepcounter{questioncounter}
    \thequestioncounter.]}{
\end{mylist}}

\newenvironment{solution}{
  \begin{trivlist}
  \item {\bf Solution.}}{
\end{trivlist}}

%=============================================================================%
               
\begin{document}

\pagestyle{plain}
\thispagestyle{fancy}
\rhead{\footnotesize Spring 2016}
\lhead{\footnotesize CS 360 Introduction to the Theory of Computing}
\cfoot{\thepage} 
\renewcommand{\headrulewidth}{0pt}
\renewcommand{\footrulewidth}{0pt}

\begin{center}
  \rule{0mm}{9mm}
  {\Large\bf Assignment 1}\\[2mm]
  Due: Friday, May 20 at 4:00pm
\end{center}

%-----------------------------------------------------------------------------%

\noindent
When answering questions on this assignment (or any other assignment or exam in
this course), you are free to make use of facts that were stated in the lecture
or that appear in the lecture notes without having to argue or reprove those
facts.

%-----------------------------------------------------------------------------%

\begin{question}[6 points]
  The following problems each have a short answer, perhaps just a few
  sentences and maybe an equation or two.
  Try to make your answers clear and to the point, and choose the simplest
  answer whenever possible.

  \begin{enumerate}
  \item[(a)]
    Let $\Sigma$ be an alphabet and let $A\subseteq\Sigma^{\ast}$ be any
    infinite language.
    Prove that there must exist a language $B\subseteq A$ that is not regular.
  \item[(b)]
    Let $\Sigma$ be an alphabet.
    Prove that there are countably many finite languages over $\Sigma$.
  \item[(c)]
    Let $\Sigma$ be an alphabet, let $M = (Q,\Sigma,\delta,q_0,F)$ be a DFA,
    and let $p\in Q$ be a state of $M$.
    Define a language
    \[
    A = \bigl\{
    w\in\Sigma^{\ast}\,:\,
    \text{when $M$ is run on input $w$ it enters the state $p$ at least once}
    \bigr\}.
    \]
    Give a precise, formal description of a DFA that recognizes the language
    $A$.

  \end{enumerate}
\end{question}

%-----------------------------------------------------------------------------%

\begin{question}[6 points]
  Let us say that a string $x$ is obtained from a string $w$ by
  \emph{deleting symbols} if it is possible to remove zero or more symbols
  from $w$ so that just the string $x$ remains.
  For example, the following strings can all be obtained from 0110 by
  deleting symbols:
  \[
  \varepsilon,\;
  0,\;
  1,\;
  00,\; 
  01,\;
  10,\;
  11,\;
  010,\; 
  011,\;
  110,\; \text{and} \; 
  0110.
  \]
  
  For the two parts of this question that follow, assume that 
  $\Sigma = \{0,1\}$ and that $A\subseteq\Sigma^{\ast}$ is a regular language.
  \begin{enumerate}
  \item[(a)]
    Prove that this language is regular:
    \[
    B = \biggl\{x\in\Sigma^{\ast}\,:\;
    \parbox[c]{2.65in}{
      there exists a string $w\in A$ such that $x$\\
      is obtained from $w$ by deleting symbols}
    \biggr\}.
    \]
  \item[(b)]
    Prove that this language is regular:
    \[
    C = \biggl\{x\in\Sigma^{\ast}\,:\;
    \parbox[c]{2.65in}{
      there exists a string $w\in A$ such that $w$\\
      is obtained from $x$ by deleting symbols}
    \biggr\}.
    \]
  \end{enumerate}
\end{question}

%-----------------------------------------------------------------------------%

\begin{question}[6 points]
  The following two questions are yes/no questions.
  In each case, answer ``yes'' or ``no,'' and give an argument in support
  of your answer.
  \begin{enumerate}
  \item[(a)]
    Let $\Sigma = \{0\}$, let $M = (Q,\Sigma,\delta,q_0,F)$ be a DFA, and
    assume that $M$ accepts every string $w\in\Sigma^{\ast}$ such that
    $\abs{w}<\abs{Q}$.
    Is it necessarily the case that $\op{L}(M) = \Sigma^{\ast}$?
  \item[(b)]
    Let $\Sigma = \{0\}$, let $N = (Q,\Sigma,\delta,q_0,F)$ be an NFA, and
    assume that $N$ accepts every string $w\in\Sigma^{\ast}$ such that
    $\abs{w}<\abs{Q}$.
    Is it necessarily the case that $\op{L}(N) = \Sigma^{\ast}$?
  \end{enumerate}
\end{question}

%-----------------------------------------------------------------------------%

\begin{question}[4 points]
  Let $\Sigma = \{0,1\}$, and define a language
  \[
  \textup{Middle} = \bigl\{u\tinyspace 1\tinyspace v\,:\,u,v\in\Sigma^{\ast}\:
  \text{and} \:|u| = |v|\bigr\}.
  \]
  In words, $\textup{Middle}$ is the language of all binary strings of odd
  length whose middle symbol is~1.
  Prove that $\textup{Middle}$ is not regular.
\end{question}

%-----------------------------------------------------------------------------%

\begin{question}[2 points]
  This is intended to be a (comparatively) difficult problem.
  It's only worth 2 points---so give it a try but don't worry if you don't solve
  it.

  Let $\Sigma$ be an alphabet and let $A\subseteq\Sigma^{\ast}$ be a regular
  language.
  Prove that the language
  \[
  B = \bigl\{w\in\Sigma^{\ast}\,:\,www\in A\bigr\}
  \]
  is regular.
\end{question}

%-----------------------------------------------------------------------------%

\begin{question}[1 point]
  For each of the questions above, list the full name of each of your 360
  classmates with whom you worked on that question.
  (If you didn't work with anyone, that is fine: just indicate that you worked
  alone.)
\end{question}

%-----------------------------------------------------------------------------%

\end{document}
