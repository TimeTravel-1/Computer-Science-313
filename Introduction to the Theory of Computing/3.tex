\documentclass[11pt]{article}

\usepackage[margin=1in]{geometry}
\usepackage{mathpazo}
\usepackage{amssymb}
\usepackage{amsmath}
\usepackage{fancyhdr}

%-----------------------------------------------------------------------------%
% Macros:
%-----------------------------------------------------------------------------%

\newcommand{\op}[1]{\operatorname{#1}}
\newcommand{\tinyspace}{\mspace{1mu}}
\newcommand{\rev}{\textup{\tiny R}}

\newcommand{\abs}[1]{\lvert #1 \rvert}
\newcommand{\bigabs}[1]{\bigl\lvert #1 \bigr\rvert}
\newcommand{\Bigabs}[1]{\Bigl\lvert #1 \Bigr\rvert}
\newcommand{\biggabs}[1]{\biggl\lvert #1 \biggr\rvert}
\newcommand{\Biggabs}[1]{\Biggl\lvert #1 \Biggr\rvert}

\renewcommand{\natural}{\mathbb{N}}

\newenvironment{mylist}[1]{\begin{list}{}{
	\setlength{\leftmargin}{#1}
	\setlength{\rightmargin}{0mm}
	\setlength{\labelsep}{2mm}
	\setlength{\labelwidth}{8mm}
	\setlength{\itemsep}{0mm}}}
	{\end{list}}

\newcounter{questioncounter}
\newenvironment{question}{
  \begin{mylist}{\parindent}
  \item[\stepcounter{questioncounter}
    \thequestioncounter.]}{
\end{mylist}}

\newenvironment{solution}{
  \begin{trivlist}
  \item {\bf Solution.}}{
\end{trivlist}}

%=============================================================================%
               
\begin{document}

\pagestyle{plain}
\thispagestyle{fancy}
\rhead{\footnotesize Spring 2016}
\lhead{\footnotesize CS 360 Introduction to the Theory of Computing}
\cfoot{\thepage} 
\renewcommand{\headrulewidth}{0pt}
\renewcommand{\footrulewidth}{0pt}

\begin{center}
  \rule{0mm}{9mm}
  {\Large\bf Assignment 3}\\[2mm]
  Due: Wednesday, June 29 at 4:00pm
\end{center}

%-----------------------------------------------------------------------------%

Most of your answers to the problems on this assignment will likely include
descriptions of Turing machines that operate in various ways.
You should aim for \emph{high-level descriptions} of these Turing machines
that explain the key algorithmic ideas of how they work, as opposed to
low-level descriptions that specify tape head movements and so on.

%-----------------------------------------------------------------------------%

\begin{question}[6 points]
  Prove that the following languages are decidable, by giving high-level
  descriptions of DTMs that decide them.
  \begin{align*}
    A & = 
    \biggl\{ \langle D_0,D_1 \rangle \,:\, 
    \parbox{3.25in}{$D_0$ and $D_1$ are DFAs,
      and there exists at least one string that is accepted by both
      $D_0$ and $D_1$}\;\biggr\},\\
    B & =
    \bigl\{\langle G\rangle\,:\,\text{$G$ is a CFG for which $\op{L}(G)$ is
      infinite}\bigr\},\\
    C & = 
    \bigl\{\langle D\rangle\,:\:
    \text{
      $D$ is a DFA, and every string accepted by
      $D$ is a palindrome}\bigr\}.
  \end{align*}
  In each case, you are to assume that a reasonable encoding scheme for the
  objects involved has been selected---and while the languages do indeed
  depend on the chosen encoding schemes, the fact that they are decidable does
  not.
\end{question}

%-----------------------------------------------------------------------------%

\begin{question}[6 points]
  Let $\Sigma$ be an alphabet, and suppose that $A\subseteq\Sigma^{\ast}$
  is Turing recognizable.
  Prove that these languages are also Turing recognizable:
  \begin{align*}
    \op{Prefix}(A) & = \bigl\{x\in\Sigma^{\ast}\,:\,
    \text{there exists $v\in\Sigma^{\ast}$ such that $xv\in A$}\bigr\},\\
    \op{Suffix}(A) & = \bigl\{x\in\Sigma^{\ast}\,:\,
    \text{there exists $u\in\Sigma^{\ast}$ such that $ux\in A$}\bigr\},\\
    \op{Substring}(A) & = \bigl\{x\in\Sigma^{\ast}\,:\,
    \text{there exists $u,v\in\Sigma^{\ast}$ such that $uxv\in A$}\bigr\}.
  \end{align*}
\end{question}

%-----------------------------------------------------------------------------%

\begin{question}[6 points]
  Let $\Sigma$ be an alphabet, and suppose that $A,B\subseteq\Sigma^{\ast}$
  are Turing-recognizable languages for which both $A\cap B$ and $A \cup B$ are
  decidable.
  Prove that $A$ is decidable.
\end{question}

%-----------------------------------------------------------------------------%

\begin{question}[6 points]
  Suppose that we have chosen an encoding scheme whereby each DTM $M$ is
  encoded as a string $\langle M \rangle$ over some alphabet $\Sigma$.
  Define two languages over $\Sigma$ as follows:
  \begin{align*}
    A & = \bigl\{ \langle M \rangle \,:\, \text{$M$ rejects the input
      $\langle M \rangle$} \bigr\},\\
    B & = \bigl\{ \langle M \rangle \,:\, \text{$M$ accepts the input
      $\langle M \rangle$} \bigr\}.
  \end{align*}
  For the sake of this problem, you should assume that the statements
  ``$M$ accepts the input $\langle M\rangle$'' and
  ``$M$ rejects the input $\langle M\rangle$'' are both false if it so happens
  that there are symbols in the string $\langle M\rangle$ that are not
  contained in the input alphabet of $M$.

  Prove that there does not exist a decidable language
  $C\subseteq\Sigma^{\ast}$ for which both of the containments $A\subseteq C$
  and $B \subseteq \overline{C}$ hold.

  Hint: it is possible to prove the required fact directly through the use of
  diagonalization.
  You are, however, free to prove the fact through whatever method you choose.
\end{question}

%-----------------------------------------------------------------------------%

\begin{question}[1 point]
  For each of the questions above, list the full name of each of your 360
  classmates with whom you worked on that question.
  (If you didn't work with anyone, that is fine: just indicate that you worked
  alone.)
\end{question}

%-----------------------------------------------------------------------------%

\end{document}
